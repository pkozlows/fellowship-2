\documentclass[12pt]{article}
\usepackage{amsmath}
\usepackage{physics}
\usepackage{tikz-feynman}
\usepackage{hyperref}
\usepackage{forest}
\author{Patryk Kozlowski}
\title{Proposal}
\date{\today}
\begin{document}
\maketitle

\section*{Outline}

The green energy transition underscores the need for the discovery of materials. Density Functional Theory (DFT) has long served as a computational workhorse for materials science by using the electron density as the fundamental quantity. However, it treats the repulsive interactions between electrons using an approximate exchange-correlation functional, leading to variable results. A potential solution is the application of Green's functions in many-body perturbation theory (MBPT). Central is the Dyson equation

\begin{equation}
\begin{aligned}
    G &= \textcolor{green}{G_0} + \textcolor{blue}{G_0 \Sigma G_0} + \textcolor{red}{G_0 \Sigma G_0 \Sigma G_0} + \ldots\\
\end{aligned}
\label{eq:dyson}
\end{equation}
\begin{equation}
  = 
\textcolor{green}{
\feynmandiagram [horizontal=a to b] {
   a -- [fermion] b,
};
}
+
\textcolor{blue}{
\feynmandiagram [horizontal=a to b] {
   a -- [fermion] c [blob] -- [fermion] b,
};
}
+
\textcolor{red}{
\feynmandiagram [horizontal=a to b] {
   a -- [fermion] c [blob] -- [fermion] d [blob] -- [fermion] b,
};
}
+ \ldots,
\label{eq:dyson_feynman}
\end{equation}


where equation \ref{eq:dyson} relates the Green's function for the fully interacting system \( G \) to that pf the non interacting system \( G_0 \) through the self-energy \( \Sigma \). In terms of the Feynman diagrams of equation \ref{eq:dyson_feynman}, \( G_0 \) is represented by a single line and \( \Sigma \) by a blob. In the \( GW \) approximation, the self-energy \( \Sigma \) takes the form \( iGW \), where \( W \) is the screened Coulomb interaction. Therefore, the Dyson equation represents a series expansion in the interaction strength \( W \), since it is used to make \( \Sigma \); the \( GW \) approximation is accurate for systems where it is reasonable to perturbatively expand the Dyson equation in the Coulomb interaction. However, for strongly correlated systems, in which the majority of current materials science research lies, the interaction is large, so the \( GW \) approximation is often poor.

There is the Mori-Zwanzig (MZ) theory of statistical physics, which has recently been applied to Green's functions. We start with the differential form of the MZ equation,
\begin{equation}
\frac{d}{d t} \mathcal{P} e^{t \mathcal{L}} \mathcal{P} = \mathcal{P} e^{t \mathcal{L}} \mathcal{P} \mathcal{L} \mathcal{P} + \int_0^t \mathcal{P} e^{(t-s) \mathcal{L}} \mathcal{P} \mathcal{L} e^{s \mathcal{Q} \mathcal{L}} \mathcal{Q} \mathcal{L} \mathcal{P} \, ds , 
\label{eq:mz}
\end{equation}
where 
$\mathcal{L}$ is the Liouville operator, \( \mathcal{U}(t, 0) = e^{t \mathcal{L}} \) is the time propagator, and \( \mathcal{P} \) is a projection operator. The projection operator is used to isolate the system of interest and the orthogonal complement \( \mathcal{Q} = \mathcal{I} - \mathcal{P} \) is treated as a bath. Equation \ref{eq:mz} can be used to derive the equation of motion for the Green's function \( G(t) \), as

\begin{equation}
\frac{d}{d t} G(t) = \Omega G(t) + \int_0^t \hat{\Sigma}(s) G(t-s) \, ds,
\end{equation}
where \( \Omega \) is the frequency response and a \( \hat{\Sigma}(s) \) is the memory colonel  analogous to the self-energy \( \Sigma \) in MBPT. A diagrammatic theory analogous to Feynman diagrams has been introduced for the MZ framework, in the form of trees. The memory kernel \( \hat{\Sigma} \) can be expanded as
\begin{equation}\label{bare_expansion_CMZE_sigma}
\begin{aligned}
\hat\Sigma
&=\textcolor{green}{\hat\Sigma^0[t]} + \textcolor{blue}{\hat\Sigma^1[t]} + \textcolor{red}{\hat\Sigma^2[t]} + \cdots \\
&=
\textcolor{green}{
\biggl[
\ldots+
{\small
\begin{forest}
for tree={grow=90, l=1mm}
[[][]]
\path[fill=green]  (!.parent anchor) circle[radius=2pt];
\path[fill=green] (!1.child anchor) circle[radius=2pt]
                 (!2.child anchor) circle[radius=2pt];
\end{forest}
}
\biggr]
}
+
\textcolor{blue}{
t\biggl[
\ldots
+
{\small
\begin{forest}
 for tree={grow=90, l=1mm}
[[][][]]
 \path[fill=blue]  (!.parent anchor) circle[radius=2pt];
 \path[fill=blue] (!1.child anchor) circle[radius=2pt]
                  (!2.child anchor) circle[radius=2pt]
                  (!3.child anchor) circle[radius=2pt];
\end{forest}
}
\biggr]
}
+
\textcolor{red}{
\frac{t^2}{2}
\Biggl[
\ldots
+
{\small
\begin{forest}
 for tree={grow=90, l=1mm}
 [ [[][]] []]
 \path[fill=red]  (!.parent anchor) circle[radius=2pt];
 \path[fill=red] (!1.child anchor) circle[radius=2pt]
                  (!11.child anchor) circle[radius=2pt]
                  (!12.child anchor) circle[radius=2pt]
                  (!2.child anchor)  circle[radius=2pt];
\end{forest}
}
\ldots
\Biggr]
}
+\ldots
\end{aligned}
\end{equation}

Equation \ref{bare_expansion_CMZE_sigma} shows the advantage over Feynman diagrams, as the expansion is made in powers of the evolution time \( t \) rather than the interaction strength \( W \). 

Taking the Laplace transform of this equation gives a result analogous to the series expansion of the Dyson equation \ref{eq:dyson}:

\begin{equation}
   G(z)=S^{-1}(z) G(0)+S^{-1}(z) \hat{\Sigma}(z) S^{-1}(z) G(0)+S^{-1}(z) \hat{\Sigma}(z) S^{-1}(z) \hat{\Sigma}(z) S^{-1}(z) G(0)+\ldots,
\end{equation}
where \( S^{-1}(z)=(z I-\Omega)^{-1} \) and the initial condition $G(0)$ replaces the non-interacting Green's function \( G_0 \).

\section*{Motivation and Intellectual Merit:}
In my senior thesis, I implemented $G_0W_0$, which is an extension of the $GW$ approximation, for molecules. This has prepared me to think about Green's functions in MBPT, now in condensed matter. In addition, I gave talks on my research (Caltech Senior Thesis and Goldwater Symposiums) and attended the BerkeleyGW conference, where I learned about the current state of the $GW$ community that I will be a part of in the future.

\section*{Research Plan}
\textbf{Aim 1}: The uniform electron gas (UEG) is a paradigmatic system in condensed matter physics. As my rotation project with Prof. Lee, I will implement fully self-consistent $GW$ (scGW) for the UEG. I am interested in whether I can corroborate the result reported in the literature where scGW predicts only one quasi-particle peak in the frequency spectrum, while Quantum Monte Carlo (QMC) shows an additional satellite peak.\\
\textbf{Aim 2}: Perform a similar investigation on the UEG in the Mori-Zwanzig framework. Assess improvement over $GW$.\\
\textbf{Aim 3}: Apply the MZ framework to a realistic condensed matter system, such as the Hubbard model.\\
\textbf{Aim 4}: During my project, I will be using various diagrammatic theories; I have a fine motor impairment resulting from a stroke, so this project will provide the motivation to improve my handwriting through practice. In addition, I will gain experience in how to typeset diagrams efficiently in LaTeX, as I do above.

\section*{Broader Impacts}
The proposed research develops the MZ theoretical framework by proposing a computational implementation of it. This will serve as an alternative to the $GW$ approximation, enabling the investigation of strongly correlated systems for the design of materials with sustainability applications. 

\end{document}