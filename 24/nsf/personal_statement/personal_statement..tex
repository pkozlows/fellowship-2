\documentclass[11pt]{article} % 11pt font

% Conform to NSF formatting requirements
% see: https://www.nsf.gov/pubs/2020/nsf20587/nsf20587.pdf
% --------------------------------------
% 1in margins
\usepackage[margin=1in]{geometry}
% use times new roman for main text
\usepackage{fontspec}
\setmainfont{Times New Roman}
% single line spacing
\usepackage{bookmark}
\usepackage{setspace}
\usepackage{amsmath}
\usepackage{physics}
\usepackage{tikz-feynman}
\usepackage{hyperref}
\usepackage{forest}
\usepackage{wrapfig}
\usepackage{enumitem}
\usepackage{graphicx}
\singlespacing

% To fit more into the proposal, 
% let's make the section titles tiny and compact
\usepackage[tiny,compact]{titlesec}
\titleformat{\section}[runin]{\bfseries}{\thesection}{1em}{}
\titleformat{\subsection}[runin]{\bfseries}{\thesubsection}{1em}{}



% -------------------------
% BEGINNING OF THE DOCUMENT
% -------------------------
\begin{document}



The Personal, Relevant Background and Future Goals Statement must address NSF's merit review criteria of Intellectual Merit and Broader Impacts. Applicants must include headings for Intellectual Merit and Broader Impacts in their statements.

The maximum length of the Personal, Relevant Background and Future Goals Statement is three (3) pages.
\section{Personal}
When awarded the Goldwater Scholarship, most people would jump out of their chair, but I couldn't move up or down in my hospital bed. It was the middle of my junior year, and I was diagnosed with leukemia and subsequently had a stroke, which left me with motor deficits. I was initially bed-bound and could not communicate. As the weeks started to pass by and the reality of a long recovery settled in, I was faced with deep questions as to what extent I wanted to continue rehabilitating. My choice to not give up was motivated by the realization that I had a bright future ahead of me. When the psychologist at the hospital would see me, I would tell her about my dreams of attending graduate school with barely understandable speech from my wheelchair. Indeed, it is magical that I am at Harvard applying for the prestigious NSF GRFP.

Returning home after four months in the hospital for acute rehabilitation, I was met with a particularly insidious combination of motor impairment and chemotherapy. Taking high-dose steroids for the first time, which caused sleep issues with an increased appetite, I would stare at the ceiling throughout the night in bed and then sit up to eat every 3 hours, even with my hands so shaky that I could barely handle a spoon. I could not go to the bathroom myself; my parents set up a call button for me to be able to ring them 2-3 times a night. Complaining about academic challenges just doesn't feel right anymore. Once these issues subsided, I took to physical rehabilitation. After 3 intense years, the improved presentation is that I use an assistive device to walk. Typing and handwriting are time-consuming due to my impaired fine motor function, and I have weakness in my articulators, a slower rate of speech, and difficulty changing pitch. To many, this would seem like the end of an academic career, but with the advent of AI, I saw an opportunity. First of all, I took the initiative to learn AI-powered dictation to code by voice. To do so, I joined the Talon Slack channel, where the community for my free dictation software meets. Improving the software is a continuous project with everybody contributing. There are regular meetups over Discord, where users can ask questions of the more experienced. Often, people stay around afterward to chat about how to improve the hands-free coding experience while sharing the screen of their personal dictation setup. When ChatGPT went viral, I had the idea of interfacing with it to correct my dictated text. Because of my speech impairment, the computer doesn't register everything perfectly, but ChatGPT solves this. For example, the computer might hear "The quick front dogs jumps over the late dog," whereas ChatGPT might change it to what it thinks I had meant to say: "The quick brown fox jumps over the lazy dog." This integration of LLMs with dictation led others to think about other uses. For example, one user thought of translating a piece of English text into Japanese. Instead of clicking around online to get a mediocre translation from Google Translate, you can make a selection and say, "Model, please translate this to Japanese," and get back a state-of-the-art translation immediately in your text editor, courtesy of ChatGPT, which is omnilingual. For most people, this means a few seconds savings, but for those disabled with difficulty using the mouse, it is priceless. All of this was to help me get back into my academic passion of quantum chemistry.
\section{Intellectual Merit}
My first contact with the field was in 2019, when I worked with Prof. Hadt in a physical inorganic chemistry laboratory at Caltech. I was developing a computational model for characterizing spin-phonon coupling in Co(III) complexes, in a fundamental study relevant to photocatalysis and quantum information science. Amidst running DFT calculations to determine vibrational modes of optimized structures and then tracking energetics with the multi-reference CASSCF, I became fascinated by this intersection of chemistry with computation. I wanted to know what was going on behind the hood of these quantum chemistry methods. Whenever my calculations were running, I would pester my graduate student mentor to recommend reading materials. I started learning electronic structure theory that summer via “Modern Quantum Chemistry” by Szabo and Ostlund.

The following year, I worked with Prof. Chan in a quantum chemistry group at Caltech. I computed surface energies of a platinum (111) surface, which is used as a heterogeneous catalyst to sustainably produce fertilizers. First, I used DFT to corroborate literature about the method's overestimation of surface stability. I became familiar with the slabs and k-point meshes that enable one to perform periodic calculations, along with the electron smearing and density fitting that are used to improve performance. Then, I investigated a newly developed periodic CCSD method through calculations on the same system. Plagued with convergence issues, I was not able to use the post Hartree-Fock CCSD to overcome the limitations of DFT. Thus, I learned about the motivation for research in quantum chemistry: going beyond the mean field of DFT to achieve accuracy in complex systems at a reasonable cost. I continued my reading of Szabo, focusing on the standard models of quantum chemistry, perturbative approaches, and Green function methods.

In Fall 2020, I worked as a TA for an introductory quantum mechanics course for chemists. Along with the other TAs, I met with Prof. Okumura weekly to discuss what aspects of the lecture students were struggling with and how we could design our recitations to address this. Throughout this quarter, I also wrote a publication describing the findings from my research of the previous summer in the Caltech Undergraduate Research Journal. I thoroughly enjoyed the challenge of scientific communication, which involves explaining complex concepts to the unfamiliar.

While on medical leave for my stroke, I implemented and optimized Full Configuration Interaction (FCI) for a simple H6 chain still with Prof. Chan. This project introduced me to dealing with large, sparse matrices in quantum chemistry. I implemented the algorithm that Ernest Davidson came up with when faced with that very same problem in the 1970s by computing just the few lowest eigenvalues of relevance of the large, sparse FCI matrix. I also learned how to circumvent expensive generation of the FCI matrix altogether, with the one-particle matrix as proposed by Handy and Knowles in 1984.

Last year, I embarked on a senior thesis project, implementing $G_0W_0$ for molecules within the $GW$ approximation of many-body perturbation theory. Relying on an fully analytic frequency integration scheme, I went on to investigate a jack-of-all-trades in the recently proposed linearized $G_0W_0$ density matrix. Its uses range from computing total energies with the Galitski-Migdal functional to improving mean field natural occupations in a dissociation process. This formative experience introduced me to concepts that will enable me to apply Green's functions to electronic structure in my PhD studies; from how to construct the self-energy with the random phase approximation of linear response theory in the one-shot $G_0W_0$ to the various levels of self-consistency (evGW, QSGW, etc.) one can introduce in both the Green's function $G$ and screened Coulomb interaction $W$. In my senior thesis, I relied on a piece of advice from my option representative Prof. Okumura. He informed me that there are two keys steps in sucessful self study: 1) you open a book and 2) you read it. A year ago, I did not know what a Green's function was, and now I find myself thinking about how to move past the GW approximation in many-body perturbation theory. There were no classes to teach me this, but if you read through enough textbooks and arXiv papers, you eventually learn what Green’s functions are about. I was faced with a contour integral in my senior thesis, but I never took complex analysis. Every day for the next few weeks, I would watch YouTube videos of people doing contour integrals while on the toilet, and eventually, you learn to do one.
\section{Future goals}  
This summer, I started my PhD in the Department of Chemistry and Chemical Biology at Harvard in a rotation with Professor Joonho Lee. Implementing Hartree-Fock for the uniform electron gas in a plane wave basis during my summer rotation has introduced me to the novel concepts of programming periodicity with C++. In the fall semester, I will build on my mean field description with fully self-consistent $GW$ (scGW) and move to the novel implementation using Mori-Zwanzig, as detailed in my research proposal.  

\section{Broader impacts}  
Harvard also gives me access to the top school in public policy, the Kennedy School, where influential people come regularly to give talks. My work applying Green's functions to solid-state systems will make the development of more efficient solar cells possible. I want to then use my expertise in photovoltaics to figure out how to get solar energy to the consumer. The policy people don't like scientists; the stereotype is that we are uninteresting people who can't communicate. I will change the narrative, getting to the Kennedy evening talks after class with my assistive device and participating with my impaired speech.

My background in rehabilitation prepares me to be part of the sustainability movement. Climate change
is going to get a whole lot worse before it gets better; there is continued resistance to climate solutions even
as disasters become more common. Not long ago, I was unable to go to the toilet by myself, but now I am
living across the country from home in Southern California independently. You learn that persistence in the
darkest moments pays off in the long run.

I am inspired by the story of Caltech Professor Frances Arnold, who suffered both the suicide of her
husband and the death of her son in an accident. She was recently awarded the Nobel Prize for her work on
protein evolution, but also oversees many corporate sustainability ventures and is the president of the Biden
Sustainability Council. One day, I too hope to turn hardship into impact.

\end{document}
